% Created 2023-03-20 Mon 23:47
% Intended LaTeX compiler: pdflatex
\documentclass[11pt]{article}
\usepackage[utf8]{inputenc}
\usepackage[T1]{fontenc}
\usepackage{graphicx}
\usepackage{longtable}
\usepackage{wrapfig}
\usepackage{rotating}
\usepackage[normalem]{ulem}
\usepackage{amsmath}
\usepackage{amssymb}
\usepackage{capt-of}
\usepackage{hyperref}
\author{Sai Nishwanth Raj Reddy\\ENG21AM3031\\6-H
        \and
        Arham Asif Syed\\ENG20AM0013\\6-G
      }
\date{\today}
\title{Automatic Door Locking System with Voice Recognition}
\hypersetup{
 pdfauthor={Sai Nishwanth Raj Reddy},
 pdftitle={Automatic Door Locking System with Voice Recognition},
 pdfkeywords={},
 pdfsubject={},
 pdfcreator={Emacs 28.2 (Org mode 9.6.1)}, 
 pdflang={English}}
\begin{document}

\maketitle


\section{Main Components}
\label{sec:orgab95d5c}
\begin{itemize}
\item Arduino
\item Bread Board
\item DC-Motor
\item Microphone
\end{itemize}

\section{Synopsis}
\label{sec:org2bd2df0}
The "Automatic Door Locking System using Voice Recognition" is an IoT project that is designed to provide a convenient and secure way to lock and unlock doors using voice commands. The project utilizes an Arduino microcontroller, a breadboard, a DC motor, and a microphone to achieve its goals.

The Arduino microcontroller is the central component of the system and is responsible for controlling the other components. It processes the voice commands from the microphone, activates the DC motor to unlock and lock the door, and manages the communication between the different components of the system.

The breadboard is used to connect the various components of the system to the Arduino microcontroller. It provides a platform for easy prototyping and testing of the system.

The DC motor is used to unlock and lock the door. The motor is connected to the locking mechanism of the door and is activated by the Arduino when a voice command is recognized.

The microphone is used to recognize the user's voice commands. The microphone is connected to the Arduino through an amplifier circuit and is capable of picking up sounds within a certain frequency range. The voice recognition module of the system compares the user's voice with a pre-stored list of authorized voices, and if the user's voice is recognized, the Arduino activates the DC motor to unlock the door.

The system works as follows: when the user approaches the door, they speak a specific voice command, which is picked up by the microphone. The Arduino then processes the voice command and sends a signal to the DC motor to unlock the door. Once the user enters the room and closes the door, the system automatically locks the door using the locking mechanism and the DC motor.

The project has several advantages. Firstly, it provides a convenient and secure way to lock and unlock doors without the need for physical keys or access cards. This makes it ideal for use in smart homes, offices, and other facilities where security is a top priority. Secondly, the system is relatively easy to build and can be customized to fit different types of doors and locking mechanisms. Thirdly, the system can be integrated with other IoT devices to provide a more comprehensive security solution.

In conclusion, the "Automatic Door Locking System using Voice Recognition" is a practical and effective IoT project that utilizes an Arduino microcontroller, a breadboard, a DC motor, and a microphone to provide a convenient and secure way to lock and unlock doors using voice commands. The project is relatively easy to build and has several advantages that make it an attractive solution for various applications.
\end{document}
